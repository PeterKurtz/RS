\documentclass[12pt]{article}

\usepackage{graphicx}
\usepackage{url}
\usepackage{hyperref}
\usepackage{verbatim}
\usepackage[title,titletoc,toc]{appendix}
\usepackage{wasysym}
\usepackage[margin=1.0in]{geometry}
\usepackage{natbib}

\begin{document}

\begin{titlepage}

\begin{center}
{\large Introduction to R} \\[1.5in]

{\LARGE \bf A Sample \LaTeX\ Document for use with RStudio} \\[.4in]
by \\[.4in]
{\bf J\"urgen Symanzik} \\[1in]
{\bf Date:} \today \\[.8in]

UTAH STATE UNIVERSITY \\[.2in]
Logan, UT \\[0.2in]
Fall 2020 \\[0.2in]
\end{center}

\thispagestyle{empty}
\vfill
\end{titlepage}


\newpage 

\pagenumbering{roman}

\tableofcontents


\newpage

\pagenumbering{arabic}


\section{Using plain \LaTeX\ via RStudio}

First, you will need to download and install  MiKTeX: 

\url{http://miktex.org/}

Now you should go into RStudio and use Tools $\rightarrow$ Global Options... $\rightarrow$ Sweave (on the left) and change the ``Weave Rnw files using:'' option (at the top) to ``knitr''. 

Notice that you can open an existing \LaTeX\ .tex file from within RStudio.
This file is recognized as a .tex file and you can translate it to pdf
via the ``Compile PDF" tab that shows up in RStudio when a .tex file is open.

Alongside the ``Compile PDF" tab in RStudio, there is a Format menu so you can write many common \LaTeX\ commands easily and you do not need to memorize them.

You will see that these formatting commands introduce \LaTeX\ code into your .tex document, such as \textbf{bold}, \emph{italic}, and \texttt{typewriter}.


If you need to do something that is more complicated, you can use one of the many references available, such as those listed in Section~\ref{docs} on page~\pageref{docs}.

For new Sections, Subsections and Sub-Subsections, you need to remove the ``*". 

\subsection*{New Subsection?}

\subsection{New Subsection}

We can easily add a skeleton bullet list:

\begin{itemize}
  \item 
\end{itemize}

which we can fill in as we like:
\begin{itemize}
  \item red
  \item yellow
  \item blue
\end{itemize}

or a numbered list:
\begin{enumerate}
  \item first
  \item second
  \item third
\end{enumerate}

or a description list:
\begin{description}
  \item[A] the first letter
  \item[B] the second letter
\end{description}


\subsection{Math Symbols}

Let's try a few math symbols: $\mu, \sigma, \lambda,
s_1 = \sum_{i=1}^n i^2, \infty, \sqrt{x}, \bar{x}$ and a \smiley. Well, that's not really a math symbol. 


Now, do the sum in a math environment:
\[
s_1 = \sum_{i=1}^n {\alpha_i}^2
\]

Notice that if we try to write words in math they don't come out right. For example, we may write $3\pi Info = 2 log(x)$. It looks better if we write $3\pi \mbox{Info} = 2 \log(x)$

If we need to write a dollar sign, we use a backslash: \$10. It's the same with percentages, such as 100\%.

Compare $\ldots$ with ...

When we want a new page, we can use the command \verb|\newpage|.


\section{ SyncTeX}

SyncTeX allows us to easily move back and forth between the .tex file (in RStudio) and the pdf viewer. If using the default Sumatra viewer, doing control-click in the .tex file takes you to the corresponding place in the pdf, while double-licking in the pdf takes you to the appropriate place in the .tex file. I wish I knew how to do this 30 years ago (but honestly, many of the supporting tools didn't exist then; so keep your eyes open for new developments in the next 30 years). 

\section{\LaTeX\ Documentation}\label{docs}

Numerous web sites provide information about \LaTeX,
including tutorials and help pages. Here are some:

\begin{itemize}
\item \url{http://www.latex-project.org/}
\item \url{http://en.wikibooks.org/wiki/LaTeX}
\item \url{http://miktex.org/}
\item \url{https://artofproblemsolving.com/wiki/index.php/LaTeX:Symbols}
\end{itemize}

If you need help on a specific latex topic, google for
\verb|latex| and the keyword or command you are interested in.


\newpage


\section{Automatic Creation of Citations and References}


\subsection{What is BibTeX?}

\url{http://www.bibtex.org/} states:
\begin{quotation}
The word ``BibTeX'' stands for a tool and a file format which are used to describe and process lists of references, mostly in conjunction with LaTeX documents.
\end{quotation}


\noindent
\url{http://www.bibtex.org/About/} further states:
\begin{quotation}
BibTeX has been widely in use since its introduction by Oren Patashnik 20 years ago. As the name suggests, it was intended to be used in combination with the typesetting system LaTeX, but it has become possible, for instance, to include BibTeX-bibliographies even in Word-Documents using third-party tools.

BibTeX utilizes a plain-text file-format which can be created and modified using an arbitrary text-editor by the user. There are tools in existence which provide a more convenient UI.
\end{quotation}

The BibTeX format is briefly described at \url{http://www.bibtex.org/Format/}.
However, it makes more sense to take a closer look at an existing
collection of BibTeX entries, e.g.
Citations.bib.


\subsection{\LaTeX\ Setup}

A common setting for publications in the statistical community is to work
with the {\it natbib} package and the {\it agsm} bibliography style.
By the way, {\it agsm} stands for {\it Australian Government Style manual}.
 
In your \LaTeX\ file, you have to make the following additions:
\begin{itemize}

\item In the preamble of your document, you have to add:
\begin{verbatim}
\usepackage{natbib}
\end{verbatim}

\item In the main document (where the references should appear), you have to add:
\begin{verbatim}
\bibliographystyle{agsm}
\bibliography{references}
\end{verbatim}
\end{itemize}



\subsection{Citations and References}

We cite references via the commands \verb|\cite{citation-key}|
or \verb|\citep{citation-key}|. The first one produces
an in--text citation of the type Author (Year). The second one
produces an in--parentheses citation of the type (Author Year).

Here are some examples produced via \verb|\cite{citation-key}|,
e.g., \cite{R} is the R citation,
\cite{Lamport85} is an excellent book about \LaTeX\, \cite{RF} is
a journal article.

We can combine several references into one cite command, e.g.,
\citep{R,Lamport85,RF}.

Moreover, we can add additional information to a citation, e.g.,
\citep[p.~5]{RF} or \citep[see for example][p.~5]{RF}.
Check the source code how this has been produced.

\subsection{Errors in the .bib File}

Errors can be easily introduced into your .bib file, e.g., by missing to
enter a single ``,'', a ``\{'', or a ``\}''. Finding these errors is not
always easy and straightforward. Unfortunately, in many cases,
errors are not even reported when translating a .bib file.

To make sure that there are no errors, you should (manually)
open the resulting .blg file in a text editor. Everything
is fine if this file looks as follows:
\begin{verbatim}
This is BibTeX, Version 0.99cThe top-level auxiliary file: RefsViaBibtex.aux
The style file: agsm.bst
Database file #1: references.bib
\end{verbatim}

If there is anything else in this file, this means that there is an 
error in your .bib file.

\subsection{Different Styles for Citations and Reference Lists}

There exist many ways how to cite references and arrange reference lists.
Some journals cite by number and do not sort the reference list 
alphabetically, but rather leave it in sequential order. As authors,
we have to follow the instructions provided by the publisher
of our article, book, or any other publication.

BibTeX makes it easy to change the appearance of citations 
and the reference list. First, comment out the 
\verb|\usepackage{natbib}| command in the preamble.
Then replace it by one of the following options:

\begin{itemize}
\item Use \verb|\usepackage[numbers]{natbib}|.
\item Use \verb|\usepackage[super]{natbib}|.
\end{itemize}

More details on the {\it natbib} package usage can be found at
\url{http://merkel.zoneo.net/Latex/natbib.php}.

A large number of different bibstyles are
documented at 
\url{http://www.cs.stir.ac.uk/~kjt/software/latex/showbst.html}.

Moreover, the commands used for citations may be called
\verb|\citeasnoun{}| or \verb|\citet{}|, to mention just some
frequently encountered modifications.

An easy--to--read BibTeX tutorial (derived from a ppt presentation) 
can be found at 
\url{http://www.ntg.nl/bijeen/pdf-s.20031113/BibTeX-tutorial.pdf}.

{\bf Whenever you change the style, you should first
remove the previous .bbl file to ensure that a new .bbl file
is created according to the new style.}


\subsection{Conclusion}

The main advantage of working with a bibliography file is that this
file is growing with your work. Over time, we are likely to cite
the same references many times. However, the appearance
of the citations and the reference list likely will differ from
document to document. There will be differences
when working on a MS report or a dissertation, a conference paper,
a journal article, or a book chapter. Likely, citation styles will be different
for different disciplines and even for different publishers.

However, once you have entered your references into your bibliography file,
you will usually only have to modify two or three commands
to adapt to a new style --- rather than manually adjusting
and reformatting dozens of references in a single document.

\bibliography{Citations}
\bibliographystyle{plain}

\end{document}

