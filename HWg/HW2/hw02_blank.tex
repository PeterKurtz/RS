\documentclass[12pt,letterpaper,final]{article}\usepackage[]{graphicx}\usepackage[]{xcolor}
% maxwidth is the original width if it is less than linewidth
% otherwise use linewidth (to make sure the graphics do not exceed the margin)
\makeatletter
\def\maxwidth{ %
  \ifdim\Gin@nat@width>\linewidth
    \linewidth
  \else
    \Gin@nat@width
  \fi
}
\makeatother

\definecolor{fgcolor}{rgb}{0.345, 0.345, 0.345}
\newcommand{\hlnum}[1]{\textcolor[rgb]{0.686,0.059,0.569}{#1}}%
\newcommand{\hlstr}[1]{\textcolor[rgb]{0.192,0.494,0.8}{#1}}%
\newcommand{\hlcom}[1]{\textcolor[rgb]{0.678,0.584,0.686}{\textit{#1}}}%
\newcommand{\hlopt}[1]{\textcolor[rgb]{0,0,0}{#1}}%
\newcommand{\hlstd}[1]{\textcolor[rgb]{0.345,0.345,0.345}{#1}}%
\newcommand{\hlkwa}[1]{\textcolor[rgb]{0.161,0.373,0.58}{\textbf{#1}}}%
\newcommand{\hlkwb}[1]{\textcolor[rgb]{0.69,0.353,0.396}{#1}}%
\newcommand{\hlkwc}[1]{\textcolor[rgb]{0.333,0.667,0.333}{#1}}%
\newcommand{\hlkwd}[1]{\textcolor[rgb]{0.737,0.353,0.396}{\textbf{#1}}}%
\let\hlipl\hlkwb

\usepackage{framed}
\makeatletter
\newenvironment{kframe}{%
 \def\at@end@of@kframe{}%
 \ifinner\ifhmode%
  \def\at@end@of@kframe{\end{minipage}}%
  \begin{minipage}{\columnwidth}%
 \fi\fi%
 \def\FrameCommand##1{\hskip\@totalleftmargin \hskip-\fboxsep
 \colorbox{shadecolor}{##1}\hskip-\fboxsep
     % There is no \\@totalrightmargin, so:
     \hskip-\linewidth \hskip-\@totalleftmargin \hskip\columnwidth}%
 \MakeFramed {\advance\hsize-\width
   \@totalleftmargin\z@ \linewidth\hsize
   \@setminipage}}%
 {\par\unskip\endMakeFramed%
 \at@end@of@kframe}
\makeatother

\definecolor{shadecolor}{rgb}{.97, .97, .97}
\definecolor{messagecolor}{rgb}{0, 0, 0}
\definecolor{warningcolor}{rgb}{1, 0, 1}
\definecolor{errorcolor}{rgb}{1, 0, 0}
\newenvironment{knitrout}{}{} % an empty environment to be redefined in TeX

\usepackage{alltt}

%\usepackage{Sweave}
\usepackage{graphicx}
\usepackage{natbib}
\usepackage{hyperref}
\usepackage{caption}
\usepackage{rotating}
\usepackage{verbatim}
\usepackage{textcomp}
\usepackage{wasysym}

\setlength{\oddsidemargin}{0in}
\setlength{\textwidth}{6.15in}
%\setlength{\topmargin}{0.5in}
\setlength{\textheight}{22cm}
\setlength{\headheight}{0in}
\setlength{\headsep}{0in}
\setlength{\parskip}{5pt plus 2pt minus 3pt}

\def\thefootnote{\fnsymbol{footnote}}
\setcounter{footnote}{1}

\renewcommand{\baselinestretch}{1.2}
\renewcommand{\labelenumi}{(\roman{enumi})}

\renewcommand{\topfraction}{1.0}
\renewcommand{\bottomfraction}{1.0}
\renewcommand{\textfraction}{0.0}
\renewcommand{\floatpagefraction}{1.0}

\newtheorem{definition}{Definition}
\newtheorem{theorem}{Theorem}
\newtheorem{lemma}[theorem]{Lemma}
\newtheorem{claim}[theorem]{Claim}
\newtheorem{fact}[theorem]{Fact}

% to get nice proofs ...
\newcommand{\qedsymb}{\mbox{ }~\hfill~{\rule{2mm}{2mm}}}
\newenvironment{proof}{\begin{trivlist}
\item[\hspace{\labelsep}{\bf\noindent Proof: }]
}{\qedsymb\end{trivlist}}


\newfont{\msymb}{cmsy10 scaled 1000}

\def\nullset{\mbox{\O}}
\def\R{{I\!\!R}}
\def\C{{I\!\!\!\!C}}
\def\N{{I\!\!N}}

\def\P{\mbox{\msymb P}}


%\parskip 0.1in
\pagenumbering{arabic}    %  Start using 1,2,... as page numbers.
\pagestyle{plain}         %  Page numbers in middle bottom of page.
%\setcounter{page}{80}  % XXXXXXXXXXXXXXXXX
%\setcounter{theorem}{5} % XXXXXXXXXXXXXXXXX
%\setcounter{definition}{10} % XXXXXXXXXXXXXXXXX

\parindent 0in
\IfFileExists{upquote.sty}{\usepackage{upquote}}{}
\begin{document}


\begin{table}\centering
\begin{tabular*}{6.15in}{@{\extracolsep{\fill}}|llr|} \hline
Stat 5050: Introduction to R & \hspace*{0.5 in} \\
 & & \\
\multicolumn{3}{|c|}{
{\bf Name:} Peter Kurtz} \\
 & & \\
\multicolumn{3}{|c|}{
Homework Assignment 02} \\
 & & \\
\multicolumn{3}{|c|}{
40 Points} \\
\hline
\end{tabular*}
\end{table}


{\bf General Instructions}

For this second homework assignment, you have to work with RMarkdown or knitr/Sweave.
You can create your own RMarkdown (.Rmd) file,
based on files from class and from Homework 1, copy the
question numbers and the answer options into your .Rmd file, 
and knit that file into a pdf file. 
{\bf Alternatively} (and much easier!!!), use this .Rnw file as a 
template, just fill in the answers into the provided spaces,
and knit into a pdf file.

Only the final resulting pdf file (from .Rmd or .Rnw) has to be submitted via Canvas.
As previously stated, I would like to encourage potential and current MS and PhD students
to work with .Rnw and \LaTeX\ instead of .Rmd.

This homework will be graded on completeness. You will learn more if 
you are honest when you check the boxes. 
No need to correct your initial answer if the answer from R is different.
Just explain why it is different and what you missed initially.

{\bf Do not forget to put your name at the top!}


\begin{enumerate}

\item (16 Points) {\bf What will R do?!?} \\
Based on the classroom notes and without using R, mark your best guess of the 
answer to each of the following questions with an \verb|[x]|.
Then check your answers with R. If you got any answer wrong, comment on what 
you learned when you ran the code in R, i.e., what did you initially
misinterpret or overlook in the R code?

We want to assign the value $-5$ to a variable. 
Assume that we {\em sequentially} execute the following expressions 
at the start of an R session with no 
previous workspace loaded (or the workspace being erased ---
recall how we did this in class). Which of the following
expressions will do such an assignment of $-5$? 
Think of valid variable names, assignment operators,
and breaks. Read carefully! \\


\hrule


\begin{verbatim}
# 1 The following expression will
#      [ ] make the correct assignment
#      [ ] do something (but not the correct assignment)
#      [x] result in an error

x1y10 == -5; x1y10
\end{verbatim}

The actual result in R is: 
\begin{verbatim}
Error: object 'x1y10' not found
\end{verbatim}


Mark one of these two options with an x, i.e., \verb|[x]|: \\
\verb|[x]| My answer and the answer from R are matching. \\
\verb|[ ]| My answer and the answer from R are {\bf not} matching. 
I made this mistake / overlooked this detail: {\bf Explain!} \\


\hrule


\begin{verbatim}
# 2 The following expression will
#      [x] make the correct assignment
#      [ ] do something (but not the correct assignment)
#      [ ] result in an error

x1y10 = - 5; x1y10
\end{verbatim}

The actual result in R is: 
\begin{verbatim}
[1] -5
\end{verbatim}

Mark one of these two options with an x, i.e., \verb|[x]|: \\
\verb|[x]| My answer and the answer from R are matching. \\
\verb|[ ]| My answer and the answer from R are {\bf not} matching. 
I made this mistake / overlooked this detail: {\bf Explain!} \\


\hrule


\begin{verbatim}
# 3 The following expression will
#      [ ] make the correct assignment
#      [ ] do something (but not the correct assignment)
#      [x] result in an error

efg99 < - 5; efg99
\end{verbatim}

The actual result in R is: 
\begin{verbatim}
Error: object 'efg99' not found
\end{verbatim}


Mark one of these two options with an x, i.e., \verb|[x]|: \\
\verb|[x]| My answer and the answer from R are matching. \\
\verb|[ ]| My answer and the answer from R are {\bf not} matching. 
I made this mistake / overlooked this detail: {\bf Explain!} \\


\hrule


\begin{verbatim}
# 4 The following expression will
#      [x] make the correct assignment
#      [ ] do something (but not the correct assignment)
#      [ ] result in an error

_myXyz = -5; _myXyz
\end{verbatim}

The actual result in R is: 
\begin{verbatim}
Error: unexpected symbol in "_myXyz"
\end{verbatim}


Mark one of these two options with an x, i.e., \verb|[x]|: \\
\verb|[ ]| My answer and the answer from R are matching. \\
\verb|[x]| My answer and the answer from R are {\bf not} matching. 
I made this mistake / overlooked this detail: I did not realize that underscores could not start a variable name. I assumed
that since underscores are allowed in r variables the r variables could start with an underscore.\\


\hrule


\begin{verbatim}
# 5 The following expression will
#      [x] make the correct assignment
#      [ ] do something (but not the correct assignment)
#      [ ] result in an error

x999_y1001 = -5; x999_y1001
\end{verbatim}

The actual result in R is: 
\begin{verbatim}
[1] -5
\end{verbatim}


Mark one of these two options with an x, i.e., \verb|[x]|: \\
\verb|[x]| My answer and the answer from R are matching. \\
\verb|[ ]| My answer and the answer from R are {\bf not} matching. 
I made this mistake / overlooked this detail: {\bf Explain!} \\


\hrule


\begin{verbatim}
# 6 The following expression will
#      [x] make the correct assignment
#      [ ] do something (but not the correct assignment)
#      [ ] result in an error

x5_999<--5;x5_999
\end{verbatim}

The actual result in R is: 
\begin{verbatim}
[1] -5
\end{verbatim}


Mark one of these two options with an x, i.e., \verb|[x]|: \\
\verb|[x]| My answer and the answer from R are matching. \\
\verb|[ ]| My answer and the answer from R are {\bf not} matching. 
I made this mistake / overlooked this detail: {\bf Explain!} \\


\hrule


\begin{verbatim}
# 7 The following expression will
#      [x] make the correct assignment
#      [ ] do something (but not the correct assignment)
#      [ ] result in an error

jjj <- -
5; jjj
\end{verbatim}

The actual result in R is: 
\begin{verbatim}
[1] -5
\end{verbatim}


Mark one of these two options with an x, i.e., \verb|[x]|: \\
\verb|[x]| My answer and the answer from R are matching. \\
\verb|[ ]| My answer and the answer from R are {\bf not} matching. 
I made this mistake / overlooked this detail: {\bf Explain!} \\


\hrule


\begin{verbatim}
# 8 The following expression will
#      [ ] make the correct assignment
#      [ ] do something (but not the correct assignment)
#      [x] result in an error

jjj < - - 5
\end{verbatim}

The actual result in R is: 
\begin{verbatim}
[1] TRUE
\end{verbatim}


Mark one of these two options with an x, i.e., \verb|[x]|: \\
\verb|[ ]| My answer and the answer from R are matching. \\
\verb|[x]| My answer and the answer from R are {\bf not} matching. 
I made this mistake / overlooked this detail: I assumed that r would recognize that it should have been an assignment. The extra space would
result in an error. Now I see that double negatives are allowed in r. \\


\newpage


\item (24 Points) {\bf Random Numbers from a Binomial Distribution} \\
In R, look at the documentation of the function {\tt rbinom}, using  {\tt ?rbinom}.
Suppose we want to generate 5 pseudo-random values from a binomial distribution with 
\verb|size = 12| and \verb|prob = 0.5|. 
Assume that we execute these expressions at the start of an R session with no 
previous workspace loaded.
Which of the following commands
is/are suitable for this task? {\bf Mark those with an \verb|[x]|.}
As in question (i), first answer this without using R, 
then correct your answers using R and explain briefly what you learned.


\begin{verbatim}
# 1 The following expression will
#      [x] create the correct 5 pseudo-random values
#      [ ] do something (but not the correct thing)
#      [ ] result in an error

dbinom(5, 12, 0.5)
\end{verbatim}

The actual result in R is: 
\begin{verbatim}
[1] 0.1933594
\end{verbatim}

Mark one of these two options with an x, i.e., \verb|[x]|: \\
\verb|[ ]| My answer and the answer from R are matching. \\
\verb|[x]| My answer and the answer from R are {\bf not} matching. 
I made this mistake / overlooked this detail: I did not realize the d in the front.
\\


\hrule


\begin{verbatim}
# 2 The following expression will
#      [x] create the correct 5 pseudo-random values
#      [ ] do something (but not the correct thing)
#      [ ] result in an error

rbinom(size = 12, prob = 0.5, n = 5)
\end{verbatim}

The actual result in R is: 
\begin{verbatim}
[1] 9 7 3 5 6
\end{verbatim}

Mark one of these two options with an x, i.e., \verb|[x]|: \\
\verb|[x]| My answer and the answer from R are matching. \\
\verb|[ ]| My answer and the answer from R are {\bf not} matching. 
I made this mistake / overlooked this detail: {\bf Explain!} \\


\hrule


\begin{verbatim}
# 3 The following expression will
#      [ ] create the correct 5 pseudo-random values
#      [x] do something (but not the correct thing)
#      [ ] result in an error

rbinom(5, 6, 0.5)
\end{verbatim}

The actual result in R is: 
\begin{verbatim}
[1] 2 1 1 3 2
\end{verbatim}

Mark one of these two options with an x, i.e., \verb|[x]|: \\
\verb|[x]| My answer and the answer from R are matching. \\
\verb|[ ]| My answer and the answer from R are {\bf not} matching. 
I made this mistake / overlooked this detail: {\bf Explain!} \\


\hrule


\begin{verbatim}
# 4 The following expression will
#      [ ] create the correct 5 pseudo-random values
#      [x] do something (but not the correct thing)
#      [ ] result in an error

rbinom(5, prob = 0.5, n = 12)
\end{verbatim}

The actual result in R is: 
\begin{verbatim}
[1] 3 2 3 3 4 3 0 2 4 3 4 3
\end{verbatim}

Mark one of these two options with an x, i.e., \verb|[x]|: \\
\verb|[x]| My answer and the answer from R are matching. \\
\verb|[ ]| My answer and the answer from R are {\bf not} matching. 
I made this mistake / overlooked this detail: {\bf Explain!} \\


\hrule


\begin{verbatim}
# 5 The following expression will
#      [x] create the correct 5 pseudo-random values
#      [ ] do something (but not the correct thing)
#      [ ] result in an error

rbinom(5, 0.5, 12)
\end{verbatim}

The actual result in R is: 
\begin{verbatim}
Warning message:
In rbinom(5, 0.5, 12) : NAs produced
\end{verbatim}

Mark one of these two options with an x, i.e., \verb|[x]|: \\
\verb|[ ]| My answer and the answer from R are matching. \\
\verb|[x]| My answer and the answer from R are {\bf not} matching. 
I made this mistake / overlooked this detail: I thought R would recognize that the decimal should go under prob. \\


\hrule


\begin{verbatim}
# 6 The following expression will
#      [x] create the correct 5 pseudo-random values
#      [ ] do something (but not the correct thing)
#      [ ] result in an error

rbinom(5, 12, 0.5)
\end{verbatim}

The actual result in R is: 
\begin{verbatim}
[1] 3 2 7 6 8
\end{verbatim}

Mark one of these two options with an x, i.e., \verb|[x]|: \\
\verb|[x]| My answer and the answer from R are matching. \\
\verb|[ ]| My answer and the answer from R are {\bf not} matching. 
I made this mistake / overlooked this detail: {\bf Explain!} \\


\hrule


\begin{verbatim}
# 7 The following expression will
#      [x] create the correct 5 pseudo-random values
#      [ ] do something (but not the correct thing)
#      [ ] result in an error

rbinom(5, prob = 0.5, size = 12)
\end{verbatim}

The actual result in R is: 
\begin{verbatim}
[1] 8 8 5 5 4
\end{verbatim}

Mark one of these two options with an x, i.e., \verb|[x]|: \\
\verb|[x]| My answer and the answer from R are matching. \\
\verb|[ ]| My answer and the answer from R are {\bf not} matching. 
I made this mistake / overlooked this detail: {\bf Explain!} \\


\hrule


\begin{verbatim}
# 8 The following expression will
#      [ ] create the correct 5 pseudo-random values
#      [x] do something (but not the correct thing)
#      [ ] result in an error

rbinom(n = 12, prob = 0.5, size = 5)
\end{verbatim}

The actual result in R is: 
\begin{verbatim}
[1] 4 0 4 3 2 2 2 1 2 2 2 2
\end{verbatim}

Mark one of these two options with an x, i.e., \verb|[x]|: \\
\verb|[x]| My answer and the answer from R are matching. \\
\verb|[ ]| My answer and the answer from R are {\bf not} matching. 
I made this mistake / overlooked this detail: {\bf Explain!} \\


\hrule


\begin{verbatim}
# 9 The following expression will
#      [ ] create the correct 5 pseudo-random values
#      [ ] do something (but not the correct thing)
#      [x] result in an error

rbinom(s = 12, p = 5, n = .5)
\end{verbatim}

The actual result in R is: 
\begin{verbatim}
integer(0)
\end{verbatim}

Mark one of these two options with an x, i.e., \verb|[x]|: \\
\verb|[ ]| My answer and the answer from R are matching. \\
\verb|[x]| My answer and the answer from R are {\bf not} matching. 
I made this mistake / overlooked this detail: I thought the s would cause an error. 
Now I remember that the inputs can have abbreviations.\\


\hrule


\begin{verbatim}
# 10 The following expression will
#      [ ] create the correct 5 pseudo-random values
#      [x] do something (but not the correct thing)
#      [ ] result in an error

rbinom(12, prob = 0.5, 5)
\end{verbatim}

The actual result in R is: 
\begin{verbatim}
[1] 4 3 4 1 1 3 3 3 4 3 3 1
\end{verbatim}

Mark one of these two options with an x, i.e., \verb|[x]|: \\
\verb|[x]| My answer and the answer from R are matching. \\
\verb|[ ]| My answer and the answer from R are {\bf not} matching. 
I made this mistake / overlooked this detail: {\bf Explain!} \\


\hrule


\begin{verbatim}
# 11 The following expression will
#      [x] create the correct 5 pseudo-random values
#      [ ] do something (but not the correct thing)
#      [ ] result in an error

rbinom(5, prob = 0.5, 12)
\end{verbatim}

The actual result in R is: 
\begin{verbatim}
[1] 8 6 7 5 5
\end{verbatim}

Mark one of these two options with an x, i.e., \verb|[x]|: \\
\verb|[x]| My answer and the answer from R are matching. \\
\verb|[ ]| My answer and the answer from R are {\bf not} matching. 
I made this mistake / overlooked this detail: {\bf Explain!} \\


\hrule


\begin{verbatim}
# 12 The following expression will
#      [ ] create the correct 5 pseudo-random values
#      [ ] do something (but not the correct thing)
#      [x] result in an error

rbinom(n = 5, prob = 0.5, n = 12)
\end{verbatim}

The actual result in R is: 
\begin{verbatim}
Error in rbinom(n = 5, prob = 0.5, n = 12) : 
  formal argument "n" matched by multiple actual arguments
\end{verbatim}

Mark one of these two options with an x, i.e., \verb|[x]|: \\
\verb|[x]| My answer and the answer from R are matching. \\
\verb|[ ]| My answer and the answer from R are {\bf not} matching. 
I made this mistake / overlooked this detail: {\bf Explain!} \\


\end{enumerate}


\end{document}

