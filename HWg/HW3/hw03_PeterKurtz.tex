\documentclass[12pt,letterpaper,final]{article}\usepackage[]{graphicx}\usepackage[]{xcolor}
% maxwidth is the original width if it is less than linewidth
% otherwise use linewidth (to make sure the graphics do not exceed the margin)
\makeatletter
\def\maxwidth{ %
  \ifdim\Gin@nat@width>\linewidth
    \linewidth
  \else
    \Gin@nat@width
  \fi
}
\makeatother

\definecolor{fgcolor}{rgb}{0.345, 0.345, 0.345}
\newcommand{\hlnum}[1]{\textcolor[rgb]{0.686,0.059,0.569}{#1}}%
\newcommand{\hlstr}[1]{\textcolor[rgb]{0.192,0.494,0.8}{#1}}%
\newcommand{\hlcom}[1]{\textcolor[rgb]{0.678,0.584,0.686}{\textit{#1}}}%
\newcommand{\hlopt}[1]{\textcolor[rgb]{0,0,0}{#1}}%
\newcommand{\hlstd}[1]{\textcolor[rgb]{0.345,0.345,0.345}{#1}}%
\newcommand{\hlkwa}[1]{\textcolor[rgb]{0.161,0.373,0.58}{\textbf{#1}}}%
\newcommand{\hlkwb}[1]{\textcolor[rgb]{0.69,0.353,0.396}{#1}}%
\newcommand{\hlkwc}[1]{\textcolor[rgb]{0.333,0.667,0.333}{#1}}%
\newcommand{\hlkwd}[1]{\textcolor[rgb]{0.737,0.353,0.396}{\textbf{#1}}}%
\let\hlipl\hlkwb

\usepackage{framed}
\makeatletter
\newenvironment{kframe}{%
 \def\at@end@of@kframe{}%
 \ifinner\ifhmode%
  \def\at@end@of@kframe{\end{minipage}}%
  \begin{minipage}{\columnwidth}%
 \fi\fi%
 \def\FrameCommand##1{\hskip\@totalleftmargin \hskip-\fboxsep
 \colorbox{shadecolor}{##1}\hskip-\fboxsep
     % There is no \\@totalrightmargin, so:
     \hskip-\linewidth \hskip-\@totalleftmargin \hskip\columnwidth}%
 \MakeFramed {\advance\hsize-\width
   \@totalleftmargin\z@ \linewidth\hsize
   \@setminipage}}%
 {\par\unskip\endMakeFramed%
 \at@end@of@kframe}
\makeatother

\definecolor{shadecolor}{rgb}{.97, .97, .97}
\definecolor{messagecolor}{rgb}{0, 0, 0}
\definecolor{warningcolor}{rgb}{1, 0, 1}
\definecolor{errorcolor}{rgb}{1, 0, 0}
\newenvironment{knitrout}{}{} % an empty environment to be redefined in TeX

\usepackage{alltt}

%\usepackage{Sweave}
\usepackage{graphicx}
\usepackage{natbib}
\usepackage{hyperref}
\usepackage{caption}
\usepackage{rotating}
\usepackage{verbatim}
\usepackage{textcomp}
\usepackage{wasysym}

\setlength{\oddsidemargin}{0in}
\setlength{\textwidth}{6.15in}
%\setlength{\topmargin}{0.5in}
\setlength{\textheight}{22cm}
\setlength{\headheight}{0in}
\setlength{\headsep}{0in}
\setlength{\parskip}{5pt plus 2pt minus 3pt}

\def\thefootnote{\fnsymbol{footnote}}
\setcounter{footnote}{1}

\renewcommand{\baselinestretch}{1.2}
\renewcommand{\labelenumi}{(\roman{enumi})}

\renewcommand{\topfraction}{1.0}
\renewcommand{\bottomfraction}{1.0}
\renewcommand{\textfraction}{0.0}
\renewcommand{\floatpagefraction}{1.0}

\newtheorem{definition}{Definition}
\newtheorem{theorem}{Theorem}
\newtheorem{lemma}[theorem]{Lemma}
\newtheorem{claim}[theorem]{Claim}
\newtheorem{fact}[theorem]{Fact}

% to get nice proofs ...
\newcommand{\qedsymb}{\mbox{ }~\hfill~{\rule{2mm}{2mm}}}
\newenvironment{proof}{\begin{trivlist}
\item[\hspace{\labelsep}{\bf\noindent Proof: }]
}{\qedsymb\end{trivlist}}


\newfont{\msymb}{cmsy10 scaled 1000}

\def\nullset{\mbox{\O}}
\def\R{{I\!\!R}}
\def\C{{I\!\!\!\!C}}
\def\N{{I\!\!N}}

\def\P{\mbox{\msymb P}}


%\parskip 0.1in
\pagenumbering{arabic}    %  Start using 1,2,... as page numbers.
\pagestyle{plain}         %  Page numbers in middle bottom of page.
%\setcounter{page}{80}  % XXXXXXXXXXXXXXXXX
%\setcounter{theorem}{5} % XXXXXXXXXXXXXXXXX
%\setcounter{definition}{10} % XXXXXXXXXXXXXXXXX

\parindent 0in
\IfFileExists{upquote.sty}{\usepackage{upquote}}{}
\begin{document}

\begin{table}\centering
\begin{tabular*}{6.15in}{@{\extracolsep{\fill}}|llr|} \hline
Stat 5050: Introduction to R & \hspace*{0.5 in}\\
 & & \\
\multicolumn{3}{|c|}{
{\bf Name:} Peter Kurtz} \\
 & & \\
\multicolumn{3}{|c|}{
Homework Assignment 03 } \\
 & & \\
\multicolumn{3}{|c|}{
40 Points} \\
\hline
\end{tabular*}
\end{table}


{\bf General Instructions}

For this third homework assignment, you have to work with RMarkdown or knitr/Sweave.
You can create your own RMarkdown (.Rmd) file,
based on files from class and from Homework 1, copy the
question numbers and the answer options into your .Rmd file, 
and knit that file into a pdf file. 
{\bf Alternatively} (and much easier!!!), use this .Rnw file as a 
template, just fill in the answers into the provided spaces,
and knit into a pdf file.

Only the final resulting pdf file (from .Rmd or .Rnw) has to be submitted via Canvas.
As previously stated, I would like to encourage potential and current MS and PhD students
to work with .Rnw and \LaTeX\ instead of .Rmd.

{\bf Do not forget to replace my name and include your name instead!}


\begin{enumerate}

\item (20 Points) {\bf Sequences in R:} \\
Use {\tt :, seq()}, or {\tt rep()} and some other operators/functions appropriately, 
but definitely {\bf not c()}, to create the following sequences:

\begin{enumerate}
\item (3 Points)
1  5  9 13 17 21 25 29 33 37 41 45 (numeric vector of length 12)

\underline{Answer:}
\begin{kframe}
\begin{alltt}
\hlkwd{seq}\hlstd{(}\hlnum{1}\hlstd{,}\hlnum{45}\hlstd{,}\hlkwc{length}\hlstd{=}\hlnum{12}\hlstd{)}
\end{alltt}
\end{kframe} [1]  1  5  9 13 17 21 25 29 33 37 41 45



\item (4 Points)
1   -20   300 -4000 50000 (numeric vector of length 5)

\underline{Answer:}
\begin{kframe}
\begin{alltt}
\hlstd{nonNeg} \hlkwb{=} \hlkwd{seq}\hlstd{(}\hlnum{1}\hlstd{,}\hlnum{5}\hlstd{)}\hlopt{*}\hlkwd{rep}\hlstd{(}\hlnum{10}\hlstd{,}\hlnum{5}\hlstd{)}\hlopt{**}\hlstd{(}\hlkwd{seq}\hlstd{(}\hlnum{0}\hlstd{,}\hlnum{4}\hlstd{))}
\hlstd{alternating}\hlkwb{=}\hlkwd{rep}\hlstd{(}\hlkwd{seq}\hlstd{(}\hlopt{-}\hlnum{1}\hlstd{,}\hlnum{1}\hlstd{,}\hlkwc{by}\hlstd{=}\hlnum{2}\hlstd{),} \hlkwc{times}\hlstd{=}\hlnum{3}\hlstd{)}\hlopt{*-}\hlnum{1}
\hlstd{nonNeg}\hlopt{*}\hlstd{alternating[}\hlopt{-}\hlnum{6}\hlstd{]}
\end{alltt}
\end{kframe}[1]     1   -20   300 -4000 50000



\item (4 Points)
``a1'' ``b2'' ``b3'' ``c4'' ``c5'' ``c6'' (character vector of length 6) \\
Note: You are allowed to use \verb|letters[...]| in this part.

\underline{Answer:}
\begin{kframe}
\begin{alltt}
\hlstd{dec}\hlkwb{=}\hlkwd{seq}\hlstd{(}\hlnum{1}\hlstd{,}\hlnum{3}\hlstd{)}
\hlstd{let} \hlkwb{=} \hlkwd{rep}\hlstd{(letters[}\hlnum{1}\hlopt{:}\hlnum{3}\hlstd{], dec)}
\hlkwd{paste}\hlstd{(let,} \hlkwd{seq}\hlstd{(}\hlnum{1}\hlstd{,}\hlnum{6}\hlstd{),} \hlkwc{sep}\hlstd{=}\hlstr{""}\hlstd{)}
\end{alltt}
\end{kframe}[1] "a1" "b2" "b3" "c4" "c5" "c6"



\item (4 Points)
FALSE TRUE FALSE TRUE FALSE TRUE (logical vector of length 6)

\underline{Answer:}
\begin{kframe}
\begin{alltt}
\hlstd{numAlt} \hlkwb{=} \hlkwd{rep}\hlstd{(}\hlkwd{seq}\hlstd{(}\hlnum{0}\hlstd{,}\hlnum{1}\hlstd{),} \hlnum{3}\hlstd{)}
\hlstd{numAlt} \hlopt \hlnum{1}
\end{alltt}
\end{kframe}[1] FALSE  TRUE FALSE  TRUE FALSE  TRUE



\item (5 Points)
 10  9  8  7  8  7  6  5  6  5  4  3  4  3  2  1 (numeric vector of length 16)

\underline{Answer:}
\begin{kframe}
\begin{alltt}
\hlkwd{rep}\hlstd{(}\hlkwd{seq}\hlstd{(}\hlnum{10}\hlstd{,}\hlnum{7}\hlstd{,}\hlkwc{by}\hlstd{=}\hlopt{-}\hlnum{1}\hlstd{),} \hlkwc{times} \hlstd{=} \hlnum{4}\hlstd{)} \hlopt{+} \hlkwd{rep}\hlstd{(}\hlkwd{seq}\hlstd{(}\hlnum{0}\hlstd{,}\hlopt{-}\hlnum{6}\hlstd{,}\hlkwc{by}\hlstd{=}\hlopt{-}\hlnum{2}\hlstd{),} \hlkwc{each} \hlstd{=} \hlnum{4}\hlstd{)}
\end{alltt}
\end{kframe} [1] 10  9  8  7  8  7  6  5  6  5  4  3  4  3  2  1


\end{enumerate}

{\bf Show your R code and the final sequence produced from within R!}

{\bf Hint:} You may want to look at the help pages (and examples!) for {\tt paste and paste0} 
how to concatenate strings. Also google how to create a letter sequence in R 
(you are not allowed to write c("a", "b", "c") in part (c) above).
Create your R expressions in multiple steps. It is unlikely that you
will succeed with your first attempt to do everything in one step immediately. If you are struggling look back at your L07 notes and exercises. 


\newpage


\item (20 Points) {\bf Mammals Data:} \\
The data set {\tt mammals} in the package {\tt MASS} contains average brain weights (g) 
and average body weights (kg) for 62 species of land mammals. 
Use R expressions or functions to find the following. 

{\bf Note:} You are not allowed to do this 
by inspecting the data set or by manually typing in names or numbers that 
have been calculated in R.

{\bf Show your R code and the final result produced from within R!}

First load the data as follows and create three new vectors:
\begin{kframe}
\begin{alltt}
\hlkwd{data}\hlstd{(mammals,} \hlkwc{package} \hlstd{=} \hlstr{"MASS"}\hlstd{)}
\hlstd{mnames} \hlkwb{<-} \hlkwd{rownames}\hlstd{(mammals)}
\hlstd{mbody} \hlkwb{<-} \hlstd{mammals}\hlopt{$}\hlstd{body}
\hlstd{mbrain} \hlkwb{<-} \hlstd{mammals}\hlopt{$}\hlstd{brain}
\end{alltt}
\end{kframe}

\begin{enumerate}
\item (2 Points)
Show the first 10 lines of the entire mammals data set.

\underline{Answer:}
\begin{knitrout}
\definecolor{shadecolor}{rgb}{0.969, 0.969, 0.969}\color{fgcolor}\begin{kframe}
\begin{alltt}
\hlkwd{head}\hlstd{(mammals,} \hlnum{10}\hlstd{)}
\end{alltt}
\begin{verbatim}
##                    body brain
## Arctic fox        3.385  44.5
## Owl monkey        0.480  15.5
## Mountain beaver   1.350   8.1
## Cow             465.000 423.0
## Grey wolf        36.330 119.5
## Goat             27.660 115.0
## Roe deer         14.830  98.2
## Guinea pig        1.040   5.5
## Verbet            4.190  58.0
## Chinchilla        0.425   6.4
\end{verbatim}
\end{kframe}
\end{knitrout}

{\bf Work with mnames, mbody, and mbrain from now on!}


\item (2 Points) 
Calculate the range (aka., maximum minus minimum) of brain weights among
the species. I want you to do the subtraction, don't just give the interval.

\underline{Answer:}
\begin{kframe}
\begin{alltt}
\hlkwd{max}\hlstd{(mbrain)} \hlopt{-} \hlkwd{min}\hlstd{(mbrain)}
\end{alltt}
\end{kframe}[1] 5711.86



\item (3 Points)
List the name of the species with the heaviest brain (but do not list the
weight for this species).

\underline{Answer:}
\begin{kframe}
\begin{alltt}
\hlstd{mnames[}\hlkwd{which} \hlstd{(mbrain} \hlopt{==} \hlkwd{max}\hlstd{(mbrain))]}
\end{alltt}
\end{kframe}[1] "African elephant"



\item (4 Points)
List the names of the species with a brain heavier than 400g and body
weight less than 500kg.

\underline{Answer:}
\begin{kframe}
\begin{alltt}
\hlstd{mnames[mbrain} \hlopt{>} \hlnum{400} \hlopt{&} \hlstd{mbody} \hlopt{<} \hlnum{500}\hlstd{]}
\end{alltt}
\end{kframe}[1] "Cow"        "Donkey"     "Gorilla"    "Human"      "Okapi"     
[6] "Chimpanzee"



\item (4 Points)
List the names of the two species with their average body weights closest
to the median of the body weights. Do not list the actual body weights 
for these two species but make sure that your result is plausible
(e.g., if an elephant is among these two species, something must be wrong).

\underline{Answer:}
\begin{kframe}
\begin{alltt}
\hlstd{x} \hlkwb{=} \hlkwd{which}\hlstd{(}\hlkwd{abs}\hlstd{(mbody} \hlopt{-} \hlkwd{median}\hlstd{(mbody))} \hlopt{==} \hlkwd{min}\hlstd{(}\hlkwd{abs}\hlstd{(mbody}\hlopt{-}\hlkwd{median}\hlstd{(mbody))))}
\hlstd{mnames[x]}
\end{alltt}
\end{kframe}[1] "Arctic fox" "Cat"       



\item (5 Points)
List the names of the top--3 species with the largest ratios between
brain and body weights (i.e., brain / body). Sort them from 
highest ratio in decreasing order.
 
\underline{Answer:}
\begin{kframe}
\begin{alltt}
\hlstd{largeRatio} \hlkwb{=} \hlstd{mbrain}\hlopt{/}\hlstd{mbody}
\hlstd{orderofRatio} \hlkwb{=} \hlkwd{order}\hlstd{(largeRatio,} \hlkwc{decreasing} \hlstd{=} \hlnum{TRUE}\hlstd{)}
\hlstd{largeVector} \hlkwb{=} \hlkwd{c}\hlstd{(orderofRatio[}\hlnum{1}\hlopt{:}\hlnum{3}\hlstd{])}
\hlstd{mnames[largeVector]}
\end{alltt}
\end{kframe}[1] "Ground squirrel"           "Owl monkey"               
[3] "Lesser short-tailed shrew"


\end{enumerate}

\end{enumerate}


\end{document}

